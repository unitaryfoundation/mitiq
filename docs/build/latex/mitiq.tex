%% Generated by Sphinx.
\def\sphinxdocclass{report}
\documentclass[letterpaper,10pt,english]{sphinxmanual}
\ifdefined\pdfpxdimen
   \let\sphinxpxdimen\pdfpxdimen\else\newdimen\sphinxpxdimen
\fi \sphinxpxdimen=.75bp\relax

\PassOptionsToPackage{warn}{textcomp}
\usepackage[utf8]{inputenc}
\ifdefined\DeclareUnicodeCharacter
% support both utf8 and utf8x syntaxes
  \ifdefined\DeclareUnicodeCharacterAsOptional
    \def\sphinxDUC#1{\DeclareUnicodeCharacter{"#1}}
  \else
    \let\sphinxDUC\DeclareUnicodeCharacter
  \fi
  \sphinxDUC{00A0}{\nobreakspace}
  \sphinxDUC{2500}{\sphinxunichar{2500}}
  \sphinxDUC{2502}{\sphinxunichar{2502}}
  \sphinxDUC{2514}{\sphinxunichar{2514}}
  \sphinxDUC{251C}{\sphinxunichar{251C}}
  \sphinxDUC{2572}{\textbackslash}
\fi
\usepackage{cmap}
\usepackage[T1]{fontenc}
\usepackage{amsmath,amssymb,amstext}
\usepackage{babel}



\usepackage{times}
\expandafter\ifx\csname T@LGR\endcsname\relax
\else
% LGR was declared as font encoding
  \substitutefont{LGR}{\rmdefault}{cmr}
  \substitutefont{LGR}{\sfdefault}{cmss}
  \substitutefont{LGR}{\ttdefault}{cmtt}
\fi
\expandafter\ifx\csname T@X2\endcsname\relax
  \expandafter\ifx\csname T@T2A\endcsname\relax
  \else
  % T2A was declared as font encoding
    \substitutefont{T2A}{\rmdefault}{cmr}
    \substitutefont{T2A}{\sfdefault}{cmss}
    \substitutefont{T2A}{\ttdefault}{cmtt}
  \fi
\else
% X2 was declared as font encoding
  \substitutefont{X2}{\rmdefault}{cmr}
  \substitutefont{X2}{\sfdefault}{cmss}
  \substitutefont{X2}{\ttdefault}{cmtt}
\fi


\usepackage[Sonny]{fncychap}
\ChNameVar{\Large\normalfont\sffamily}
\ChTitleVar{\Large\normalfont\sffamily}
\usepackage{sphinx}

\fvset{fontsize=\small}
\usepackage{geometry}


% Include hyperref last.
\usepackage{hyperref}
% Fix anchor placement for figures with captions.
\usepackage{hypcap}% it must be loaded after hyperref.
% Set up styles of URL: it should be placed after hyperref.
\urlstyle{same}
\addto\captionsenglish{\renewcommand{\contentsname}{Contents:}}

\usepackage{sphinxmessages}
\setcounter{tocdepth}{1}



\title{mitiq}
\date{Mar 22, 2020}
\release{0.1.0}
\author{Tech Team @ Unitary Fund}
\newcommand{\sphinxlogo}{\vbox{}}
\renewcommand{\releasename}{Release}
\makeindex
\begin{document}

\ifdefined\shorthandoff
  \ifnum\catcode`\=\string=\active\shorthandoff{=}\fi
  \ifnum\catcode`\"=\active\shorthandoff{"}\fi
\fi

\pagestyle{empty}
\sphinxmaketitle
\pagestyle{plain}
\sphinxtableofcontents
\pagestyle{normal}
\phantomsection\label{\detokenize{index::doc}}



\chapter{Change Log}
\label{\detokenize{changelog:change-log}}\label{\detokenize{changelog:changelog}}\label{\detokenize{changelog::doc}}

\section{Version 0.1.0 (Date)}
\label{\detokenize{changelog:version-0-1-0-date}}\begin{itemize}
\item {} 
\sphinxstylestrong{Initial release.}

\end{itemize}


\chapter{Users Guide}
\label{\detokenize{guide/guide:users-guide}}\label{\detokenize{guide/guide:guide}}\label{\detokenize{guide/guide::doc}}

\section{Overview of mitiq}
\label{\detokenize{guide/guide-overview:overview-of-mitiq}}\label{\detokenize{guide/guide-overview::doc}}
Welcome to \sphinxtitleref{mitiq} Users Guide.

Mitiq is an open source toolkit for implementing error mitigation techniques on most current intermediate\sphinxhyphen{}scale quantum computers.

The library allows to postprocess results from quantum circuits with both analog and digital techniques, interfacing with a variety of quantum circuit libraries.


\section{Zero Noise Extrapolation}
\label{\detokenize{guide/guide-zne:zero-noise-extrapolation}}\label{\detokenize{guide/guide-zne::doc}}

\subsection{Introduction}
\label{\detokenize{guide/guide-zne:introduction}}
Zero noise extrapolation (ZNE) was introduced concurrently in Ref. {[}1{]} and {[}2{]}.
With \sphinxtitleref{mitiq.zne} module it is possible to extrapolate what the expected value would be without noise. This is done by first setting up one of the key objects in \sphinxtitleref{mitiq}, which is a \sphinxcode{\sphinxupquote{mitiq.Factory}} object.


\subsection{Importing Quantum Circuits}
\label{\detokenize{guide/guide-zne:importing-quantum-circuits}}
\sphinxtitleref{mitiq} allows one to flexibly import and export quantum circuits from other libraries. Here is an example:

\begin{sphinxVerbatim}[commandchars=\\\{\}]
\PYG{g+gp}{\PYGZgt{}\PYGZgt{}\PYGZgt{} }\PYG{k+kn}{from} \PYG{n+nn}{mitiq} \PYG{k+kn}{import} \PYG{n}{Factory}
\end{sphinxVerbatim}


\chapter{API\sphinxhyphen{}doc}
\label{\detokenize{apidoc:module-mitiq}}\label{\detokenize{apidoc:api-doc}}\label{\detokenize{apidoc::doc}}\index{mitiq (module)@\spxentry{mitiq}\spxextra{module}}
This is the top level module from which functions and classes of
Mitiq can be directly imported.
\index{version() (in module mitiq)@\spxentry{version()}\spxextra{in module mitiq}}

\begin{fulllineitems}
\phantomsection\label{\detokenize{apidoc:mitiq.version}}\pysiglinewithargsret{\sphinxcode{\sphinxupquote{mitiq.}}\sphinxbfcode{\sphinxupquote{version}}}{}{}
Returns the Mitiq version number.

\end{fulllineitems}



\section{Factories}
\label{\detokenize{apidoc:module-mitiq.factories}}\label{\detokenize{apidoc:factories}}\index{mitiq.factories (module)@\spxentry{mitiq.factories}\spxextra{module}}
Contains all the main classes corresponding to different zero\sphinxhyphen{}noise extrapolation methods.
\index{BatchedFactory (class in mitiq.factories)@\spxentry{BatchedFactory}\spxextra{class in mitiq.factories}}

\begin{fulllineitems}
\phantomsection\label{\detokenize{apidoc:mitiq.factories.BatchedFactory}}\pysiglinewithargsret{\sphinxbfcode{\sphinxupquote{class }}\sphinxcode{\sphinxupquote{mitiq.factories.}}\sphinxbfcode{\sphinxupquote{BatchedFactory}}}{\emph{scalars: Iterable{[}float{]}}}{}
Abstract class of a non\sphinxhyphen{}adaptive Factory.

This is initialized with a given batch of scaling factors ("scalars").
The "self.next" method trivially iterates over the elements of "scalars"
in a non\sphinxhyphen{}adaptive way.
Convergence is achieved when all the correpsonding expectation values have been measured.

Specific (non\sphinxhyphen{}adaptive) zero\sphinxhyphen{}noise extrapolation algorithms can be derived from this class by
overriding the "self.reduce" and (if necessary) the "\_\_init\_\_" method.
\index{is\_converged() (mitiq.factories.BatchedFactory method)@\spxentry{is\_converged()}\spxextra{mitiq.factories.BatchedFactory method}}

\begin{fulllineitems}
\phantomsection\label{\detokenize{apidoc:mitiq.factories.BatchedFactory.is_converged}}\pysiglinewithargsret{\sphinxbfcode{\sphinxupquote{is\_converged}}}{}{{ $\rightarrow$ bool}}
Returns True if all needed expectation values have been computed, else False.

\end{fulllineitems}

\index{next() (mitiq.factories.BatchedFactory method)@\spxentry{next()}\spxextra{mitiq.factories.BatchedFactory method}}

\begin{fulllineitems}
\phantomsection\label{\detokenize{apidoc:mitiq.factories.BatchedFactory.next}}\pysiglinewithargsret{\sphinxbfcode{\sphinxupquote{next}}}{}{{ $\rightarrow$ float}}
Returns the next noise level to execute a circuit at.

\end{fulllineitems}


\end{fulllineitems}

\index{ExpFactory (class in mitiq.factories)@\spxentry{ExpFactory}\spxextra{class in mitiq.factories}}

\begin{fulllineitems}
\phantomsection\label{\detokenize{apidoc:mitiq.factories.ExpFactory}}\pysiglinewithargsret{\sphinxbfcode{\sphinxupquote{class }}\sphinxcode{\sphinxupquote{mitiq.factories.}}\sphinxbfcode{\sphinxupquote{ExpFactory}}}{\emph{scalars: Iterable{[}float{]}, asymptote: Optional{[}float{]} = None}}{}
Factory object implementing a zero\sphinxhyphen{}noise extrapolation algotrithm assuming an
exponential ansatz y(x) = a + b * exp(\sphinxhyphen{}c * x), with c \textgreater{} 0.

If the asymptotic value (y(x\sphinxhyphen{}\textgreater{}inf) = a) is known, a linear fit with respect
to z(x) := log{[}sing(b) (y(x) \sphinxhyphen{} a){]} is used.
Otherwise, a non\sphinxhyphen{}linear fit of y(x) is perfomed.
\index{reduce() (mitiq.factories.ExpFactory method)@\spxentry{reduce()}\spxextra{mitiq.factories.ExpFactory method}}

\begin{fulllineitems}
\phantomsection\label{\detokenize{apidoc:mitiq.factories.ExpFactory.reduce}}\pysiglinewithargsret{\sphinxbfcode{\sphinxupquote{reduce}}}{}{{ $\rightarrow$ float}}
Returns the zero\sphinxhyphen{}noise limit, assuming an exponential ansatz:
y(x) = a + b * exp(\sphinxhyphen{}c * x), with c \textgreater{} 0.

\end{fulllineitems}


\end{fulllineitems}

\index{Factory (class in mitiq.factories)@\spxentry{Factory}\spxextra{class in mitiq.factories}}

\begin{fulllineitems}
\phantomsection\label{\detokenize{apidoc:mitiq.factories.Factory}}\pysigline{\sphinxbfcode{\sphinxupquote{class }}\sphinxcode{\sphinxupquote{mitiq.factories.}}\sphinxbfcode{\sphinxupquote{Factory}}}
Abstract class designed to adaptively produce a new noise scaling parameter
based on a historical stack of previous noise scale parameters ("self.instack")
and previously estimated expectation values ("self.outstack").

Specific zero\sphinxhyphen{}noise extrapolation algorithms, adaptive or non\sphinxhyphen{}adaptive,
are derived from this class.
A Factory object is not supposed to directly perform any quantum computation,
only the classical results of quantum experiments are processed by it.
\index{is\_converged() (mitiq.factories.Factory method)@\spxentry{is\_converged()}\spxextra{mitiq.factories.Factory method}}

\begin{fulllineitems}
\phantomsection\label{\detokenize{apidoc:mitiq.factories.Factory.is_converged}}\pysiglinewithargsret{\sphinxbfcode{\sphinxupquote{is\_converged}}}{}{{ $\rightarrow$ bool}}
Returns True if all needed expectation values have been computed, else False.

\end{fulllineitems}

\index{next() (mitiq.factories.Factory method)@\spxentry{next()}\spxextra{mitiq.factories.Factory method}}

\begin{fulllineitems}
\phantomsection\label{\detokenize{apidoc:mitiq.factories.Factory.next}}\pysiglinewithargsret{\sphinxbfcode{\sphinxupquote{next}}}{}{{ $\rightarrow$ float}}
Returns the next noise level to execute a circuit at.

\end{fulllineitems}

\index{push() (mitiq.factories.Factory method)@\spxentry{push()}\spxextra{mitiq.factories.Factory method}}

\begin{fulllineitems}
\phantomsection\label{\detokenize{apidoc:mitiq.factories.Factory.push}}\pysiglinewithargsret{\sphinxbfcode{\sphinxupquote{push}}}{\emph{instack\_val: float}, \emph{outstack\_val: float}}{{ $\rightarrow$ None}}
Appends "instack\_val" to "self.instack" and "outstack\_val" to "self.outstack".
Each time a new expectation value is computed this method should be used
to update the internal state of the Factory.

\end{fulllineitems}

\index{reduce() (mitiq.factories.Factory method)@\spxentry{reduce()}\spxextra{mitiq.factories.Factory method}}

\begin{fulllineitems}
\phantomsection\label{\detokenize{apidoc:mitiq.factories.Factory.reduce}}\pysiglinewithargsret{\sphinxbfcode{\sphinxupquote{reduce}}}{}{{ $\rightarrow$ float}}
Returns the extrapolation to the zero\sphinxhyphen{}noise limit.

\end{fulllineitems}


\end{fulllineitems}

\index{LinearFactory (class in mitiq.factories)@\spxentry{LinearFactory}\spxextra{class in mitiq.factories}}

\begin{fulllineitems}
\phantomsection\label{\detokenize{apidoc:mitiq.factories.LinearFactory}}\pysiglinewithargsret{\sphinxbfcode{\sphinxupquote{class }}\sphinxcode{\sphinxupquote{mitiq.factories.}}\sphinxbfcode{\sphinxupquote{LinearFactory}}}{\emph{scalars: Iterable{[}float{]}}}{}
Factory object implementing a zero\sphinxhyphen{}noise extrapolation algotrithm based on a linear fit.
\index{reduce() (mitiq.factories.LinearFactory method)@\spxentry{reduce()}\spxextra{mitiq.factories.LinearFactory method}}

\begin{fulllineitems}
\phantomsection\label{\detokenize{apidoc:mitiq.factories.LinearFactory.reduce}}\pysiglinewithargsret{\sphinxbfcode{\sphinxupquote{reduce}}}{}{{ $\rightarrow$ float}}
Determines, with a least squared method, the line of best fit
associated to the data points. The intercept is returned.

\end{fulllineitems}


\end{fulllineitems}

\index{PolyExpFactory (class in mitiq.factories)@\spxentry{PolyExpFactory}\spxextra{class in mitiq.factories}}

\begin{fulllineitems}
\phantomsection\label{\detokenize{apidoc:mitiq.factories.PolyExpFactory}}\pysiglinewithargsret{\sphinxbfcode{\sphinxupquote{class }}\sphinxcode{\sphinxupquote{mitiq.factories.}}\sphinxbfcode{\sphinxupquote{PolyExpFactory}}}{\emph{scalars: Iterable{[}float{]}, order: int, asymptote: Optional{[}float{]} = None}}{}
Factory object implementing a zero\sphinxhyphen{}noise extrapolation algotrithm assuming an
(almost) exponential ansatz with a non linear exponent, i.e.:

y(x) = a + s * exp(z(x)), where z(x) is a polynomial of a given order.

The parameter "s" is a sign variable which can be either 1 or \sphinxhyphen{}1, corresponding to
decreasing and increasing exponentials, respectively. The parameter "s" is
automatically deduced from the data.

If the asymptotic value (y(x\sphinxhyphen{}\textgreater{}inf) = a) is known, a linear fit with respect
to z(x) := log{[}s(y(x) \sphinxhyphen{} a){]} is used.
Otherwise, a non\sphinxhyphen{}linear fit of y(x) is perfomed.
\index{reduce() (mitiq.factories.PolyExpFactory method)@\spxentry{reduce()}\spxextra{mitiq.factories.PolyExpFactory method}}

\begin{fulllineitems}
\phantomsection\label{\detokenize{apidoc:mitiq.factories.PolyExpFactory.reduce}}\pysiglinewithargsret{\sphinxbfcode{\sphinxupquote{reduce}}}{}{{ $\rightarrow$ float}}
Returns the zero\sphinxhyphen{}noise limit, assuming an exponential ansatz:
y(x) = a + s * exp(z(x)), where z(x) is a polynomial of a given order.
The parameter "s" is a sign variable which can be either 1 or \sphinxhyphen{}1, corresponding to
decreasing and increasing exponentials, respectively. The parameter "s" is
automatically deduced from the data.
It is also assumed that z(x\sphinxhyphen{}\sphinxhyphen{}\textgreater{}inf)=\sphinxhyphen{}inf, such that y(x\sphinxhyphen{}\sphinxhyphen{}\textgreater{}inf)\sphinxhyphen{}\sphinxhyphen{}\textgreater{}a.

\end{fulllineitems}

\index{static\_reduce() (mitiq.factories.PolyExpFactory static method)@\spxentry{static\_reduce()}\spxextra{mitiq.factories.PolyExpFactory static method}}

\begin{fulllineitems}
\phantomsection\label{\detokenize{apidoc:mitiq.factories.PolyExpFactory.static_reduce}}\pysiglinewithargsret{\sphinxbfcode{\sphinxupquote{static }}\sphinxbfcode{\sphinxupquote{static\_reduce}}}{\emph{instack: List{[}float{]}, outstack: List{[}float{]}, asymptote: Optional{[}float{]}, order: int, eps: float = 1e\sphinxhyphen{}09}}{{ $\rightarrow$ float}}
Determines the zero\sphinxhyphen{}noise limit, assuming an exponential ansatz:
y(x) = a + s * exp(z(x)), where z(x) is a polynomial of a given order.

The parameter "s" is a sign variable which can be either 1 or \sphinxhyphen{}1, corresponding to
decreasing and increasing exponentials, respectively. The parameter "s" is
automatically deduced from the data.

It is also assumed that z(x\sphinxhyphen{}\sphinxhyphen{}\textgreater{}inf)=\sphinxhyphen{}inf, such that y(x\sphinxhyphen{}\sphinxhyphen{}\textgreater{}inf)\sphinxhyphen{}\sphinxhyphen{}\textgreater{}a.

If asymptote is None, the ansatz y(x) is fitted with a non\sphinxhyphen{}linear optimization.
Otherwise, a linear fit with respect to z(x) := log(sign * (y(x) \sphinxhyphen{} asymptote))
is performed.

This static method is equivalent to the "self.reduce" method of PolyExpFactory,
but can be called also by other factories which are particular cases of PolyExpFactory,
e.g., ExpFactory.
\begin{quote}\begin{description}
\item[{Parameters}] \leavevmode\begin{itemize}
\item {} 
\sphinxstyleliteralstrong{\sphinxupquote{instack}} \sphinxhyphen{}\sphinxhyphen{} x data values.

\item {} 
\sphinxstyleliteralstrong{\sphinxupquote{outstack}} \sphinxhyphen{}\sphinxhyphen{} y data values.

\item {} 
\sphinxstyleliteralstrong{\sphinxupquote{asymptote}} \sphinxhyphen{}\sphinxhyphen{} y(x\sphinxhyphen{}\textgreater{}inf).

\item {} 
\sphinxstyleliteralstrong{\sphinxupquote{order}} \sphinxhyphen{}\sphinxhyphen{} extrapolation order.

\item {} 
\sphinxstyleliteralstrong{\sphinxupquote{eps}} \sphinxhyphen{}\sphinxhyphen{} epsilon to regularize log(sign (instack \sphinxhyphen{} asymptote)) when
the argument is to close to zero or negative.

\end{itemize}

\end{description}\end{quote}

\end{fulllineitems}


\end{fulllineitems}

\index{PolyFactory (class in mitiq.factories)@\spxentry{PolyFactory}\spxextra{class in mitiq.factories}}

\begin{fulllineitems}
\phantomsection\label{\detokenize{apidoc:mitiq.factories.PolyFactory}}\pysiglinewithargsret{\sphinxbfcode{\sphinxupquote{class }}\sphinxcode{\sphinxupquote{mitiq.factories.}}\sphinxbfcode{\sphinxupquote{PolyFactory}}}{\emph{scalars: Iterable{[}float{]}, order: int}}{}
Factory object implementing a zero\sphinxhyphen{}noise extrapolation algotrithm based on a polynomial fit.
Note: RichardsonFactory and LinearFactory are special cases of PolyFactory.
\index{reduce() (mitiq.factories.PolyFactory method)@\spxentry{reduce()}\spxextra{mitiq.factories.PolyFactory method}}

\begin{fulllineitems}
\phantomsection\label{\detokenize{apidoc:mitiq.factories.PolyFactory.reduce}}\pysiglinewithargsret{\sphinxbfcode{\sphinxupquote{reduce}}}{}{{ $\rightarrow$ float}}
Determines with a least squared method, the polynomial of degree equal to "self.order"
which optimally fits the input data. The zero\sphinxhyphen{}noise limit is returned.

\end{fulllineitems}

\index{static\_reduce() (mitiq.factories.PolyFactory static method)@\spxentry{static\_reduce()}\spxextra{mitiq.factories.PolyFactory static method}}

\begin{fulllineitems}
\phantomsection\label{\detokenize{apidoc:mitiq.factories.PolyFactory.static_reduce}}\pysiglinewithargsret{\sphinxbfcode{\sphinxupquote{static }}\sphinxbfcode{\sphinxupquote{static\_reduce}}}{\emph{instack: List{[}float{]}, outstack: List{[}float{]}, order: int}}{{ $\rightarrow$ float}}
Determines with a least squared method, the polynomial of degree equal to \textquotesingle{}order\textquotesingle{}
which optimally fits the input data. The zero\sphinxhyphen{}noise limit is returned.

This static method is equivalent to the "self.reduce" method of PolyFactory, but
can be called also by other factories which are particular cases of PolyFactory,
e.g., LinearFactory and RichardsonFactory.

\end{fulllineitems}


\end{fulllineitems}

\index{RichardsonFactory (class in mitiq.factories)@\spxentry{RichardsonFactory}\spxextra{class in mitiq.factories}}

\begin{fulllineitems}
\phantomsection\label{\detokenize{apidoc:mitiq.factories.RichardsonFactory}}\pysiglinewithargsret{\sphinxbfcode{\sphinxupquote{class }}\sphinxcode{\sphinxupquote{mitiq.factories.}}\sphinxbfcode{\sphinxupquote{RichardsonFactory}}}{\emph{scalars: Iterable{[}float{]}}}{}
Factory object implementing Richardson\textquotesingle{}s extrapolation.
\index{reduce() (mitiq.factories.RichardsonFactory method)@\spxentry{reduce()}\spxextra{mitiq.factories.RichardsonFactory method}}

\begin{fulllineitems}
\phantomsection\label{\detokenize{apidoc:mitiq.factories.RichardsonFactory.reduce}}\pysiglinewithargsret{\sphinxbfcode{\sphinxupquote{reduce}}}{}{{ $\rightarrow$ float}}
Returns the Richardson\textquotesingle{}s extrapolation to the zero\sphinxhyphen{}noise limit.

\end{fulllineitems}


\end{fulllineitems}



\section{Zero Noise Extrapolation}
\label{\detokenize{apidoc:module-mitiq.zne}}\label{\detokenize{apidoc:zero-noise-extrapolation}}\index{mitiq.zne (module)@\spxentry{mitiq.zne}\spxextra{module}}
Zero\sphinxhyphen{}noise extrapolation tools.
\index{execute\_with\_zne() (in module mitiq.zne)@\spxentry{execute\_with\_zne()}\spxextra{in module mitiq.zne}}

\begin{fulllineitems}
\phantomsection\label{\detokenize{apidoc:mitiq.zne.execute_with_zne}}\pysiglinewithargsret{\sphinxcode{\sphinxupquote{mitiq.zne.}}\sphinxbfcode{\sphinxupquote{execute\_with\_zne}}}{\emph{qp: Union{[}qiskit.circuit.quantumcircuit.QuantumCircuit, pyquil.quil.Program{]}, executor: Callable{[}{[}Union{[}qiskit.circuit.quantumcircuit.QuantumCircuit, pyquil.quil.Program{]}{]}, float{]}, fac: mitiq.factories.Factory = None, scale\_noise: Callable{[}{[}Union{[}qiskit.circuit.quantumcircuit.QuantumCircuit, pyquil.quil.Program{]}, float{]}, Union{[}qiskit.circuit.quantumcircuit.QuantumCircuit, pyquil.quil.Program{]}{]} = None}}{{ $\rightarrow$ Callable{[}{[}Union{[}qiskit.circuit.quantumcircuit.QuantumCircuit, pyquil.quil.Program{]}{]}, float{]}}}
Takes as input a quantum circuit and returns the associated expectation value
evaluated with error mitigation.
\begin{quote}\begin{description}
\item[{Parameters}] \leavevmode\begin{itemize}
\item {} 
\sphinxstyleliteralstrong{\sphinxupquote{qp}} \sphinxhyphen{}\sphinxhyphen{} Quantum circuit to execute with error mitigation.

\item {} 
\sphinxstyleliteralstrong{\sphinxupquote{executor}} \sphinxhyphen{}\sphinxhyphen{} Function executing a circuit and producing an expectation value
(without error mitigation).

\item {} 
\sphinxstyleliteralstrong{\sphinxupquote{fac}} \sphinxhyphen{}\sphinxhyphen{} Factory object determining the zero\sphinxhyphen{}noise extrapolation algorithm.
If not specified, LinearFactory({[}1.0, 2.0{]}) will be used.

\item {} 
\sphinxstyleliteralstrong{\sphinxupquote{scale\_noise}} \sphinxhyphen{}\sphinxhyphen{} Function for scaling the noise of a quantum circuit.
If not specified, a default method will be used.

\end{itemize}

\end{description}\end{quote}

\end{fulllineitems}

\index{mitigate\_executor() (in module mitiq.zne)@\spxentry{mitigate\_executor()}\spxextra{in module mitiq.zne}}

\begin{fulllineitems}
\phantomsection\label{\detokenize{apidoc:mitiq.zne.mitigate_executor}}\pysiglinewithargsret{\sphinxcode{\sphinxupquote{mitiq.zne.}}\sphinxbfcode{\sphinxupquote{mitigate\_executor}}}{\emph{executor: Callable{[}{[}Union{[}qiskit.circuit.quantumcircuit.QuantumCircuit, pyquil.quil.Program{]}{]}, float{]}, fac: mitiq.factories.Factory = None, scale\_noise: Callable{[}{[}Union{[}qiskit.circuit.quantumcircuit.QuantumCircuit, pyquil.quil.Program{]}, float{]}, Union{[}qiskit.circuit.quantumcircuit.QuantumCircuit, pyquil.quil.Program{]}{]} = None}}{{ $\rightarrow$ Callable{[}{[}Union{[}qiskit.circuit.quantumcircuit.QuantumCircuit, pyquil.quil.Program{]}{]}, float{]}}}
Takes as input a generic function ("executor"), difined by the user, which executes a circuit
with an arbitrary backend and produces an expectation value.

Returns an error\sphinxhyphen{}mitigated version of the input "executor", having the same signature and
automatically performing zero\sphinxhyphen{}noise extrapolation at each call.
\begin{quote}\begin{description}
\item[{Parameters}] \leavevmode\begin{itemize}
\item {} 
\sphinxstyleliteralstrong{\sphinxupquote{executor}} \sphinxhyphen{}\sphinxhyphen{} Function (to be mitigated) executing a circuit and returning an expectation value.

\item {} 
\sphinxstyleliteralstrong{\sphinxupquote{fac}} \sphinxhyphen{}\sphinxhyphen{} Factory object determining the zero\sphinxhyphen{}noise extrapolation algorithm.
If not specified, LinearFactory({[}1.0, 2.0{]}) is used.

\item {} 
\sphinxstyleliteralstrong{\sphinxupquote{scale\_noise}} \sphinxhyphen{}\sphinxhyphen{} Function for scaling the noise of a quantum circuit.
If not specified, a default method is used.

\end{itemize}

\end{description}\end{quote}

\end{fulllineitems}

\index{qrun\_factory() (in module mitiq.zne)@\spxentry{qrun\_factory()}\spxextra{in module mitiq.zne}}

\begin{fulllineitems}
\phantomsection\label{\detokenize{apidoc:mitiq.zne.qrun_factory}}\pysiglinewithargsret{\sphinxcode{\sphinxupquote{mitiq.zne.}}\sphinxbfcode{\sphinxupquote{qrun\_factory}}}{\emph{fac: mitiq.factories.Factory, qp: Union{[}qiskit.circuit.quantumcircuit.QuantumCircuit, pyquil.quil.Program{]}, executor: Callable{[}{[}Union{[}qiskit.circuit.quantumcircuit.QuantumCircuit, pyquil.quil.Program{]}{]}, float{]}, scale\_noise: Callable{[}{[}Union{[}qiskit.circuit.quantumcircuit.QuantumCircuit, pyquil.quil.Program{]}, float{]}, Union{[}qiskit.circuit.quantumcircuit.QuantumCircuit, pyquil.quil.Program{]}{]}}}{{ $\rightarrow$ None}}
Runs the factory until convergence executing quantum circuits with different noise levels.
\begin{quote}\begin{description}
\item[{Parameters}] \leavevmode\begin{itemize}
\item {} 
\sphinxstyleliteralstrong{\sphinxupquote{fac}} \sphinxhyphen{}\sphinxhyphen{} Factory object to run until convergence.

\item {} 
\sphinxstyleliteralstrong{\sphinxupquote{qp}} \sphinxhyphen{}\sphinxhyphen{} Circuit to mitigate.

\item {} 
\sphinxstyleliteralstrong{\sphinxupquote{executor}} \sphinxhyphen{}\sphinxhyphen{} Function which executes a circuit and returns an expectation value.

\item {} 
\sphinxstyleliteralstrong{\sphinxupquote{scale\_noise}} \sphinxhyphen{}\sphinxhyphen{} Function which scales the noise level of a quantum circuit.

\end{itemize}

\end{description}\end{quote}

\end{fulllineitems}

\index{run\_factory() (in module mitiq.zne)@\spxentry{run\_factory()}\spxextra{in module mitiq.zne}}

\begin{fulllineitems}
\phantomsection\label{\detokenize{apidoc:mitiq.zne.run_factory}}\pysiglinewithargsret{\sphinxcode{\sphinxupquote{mitiq.zne.}}\sphinxbfcode{\sphinxupquote{run\_factory}}}{\emph{fac: mitiq.factories.Factory, noise\_to\_expval: Callable{[}{[}float{]}, float{]}, max\_iterations: int = 100}}{{ $\rightarrow$ None}}
Runs a factory until convergence (or until the number of iterations reach "max\_iterations").
\begin{quote}\begin{description}
\item[{Parameters}] \leavevmode\begin{itemize}
\item {} 
\sphinxstyleliteralstrong{\sphinxupquote{fac}} \sphinxhyphen{}\sphinxhyphen{} Instance of Factory object to be run.

\item {} 
\sphinxstyleliteralstrong{\sphinxupquote{noise\_to\_expval}} \sphinxhyphen{}\sphinxhyphen{} Function mapping noise scale values to expectation vales.

\item {} 
\sphinxstyleliteralstrong{\sphinxupquote{max\_iterations}} \sphinxhyphen{}\sphinxhyphen{} Maximum number of iterations (optional). Default value is 100.

\end{itemize}

\end{description}\end{quote}

\end{fulllineitems}

\index{zne\_decorator() (in module mitiq.zne)@\spxentry{zne\_decorator()}\spxextra{in module mitiq.zne}}

\begin{fulllineitems}
\phantomsection\label{\detokenize{apidoc:mitiq.zne.zne_decorator}}\pysiglinewithargsret{\sphinxcode{\sphinxupquote{mitiq.zne.}}\sphinxbfcode{\sphinxupquote{zne\_decorator}}}{\emph{fac: mitiq.factories.Factory = None, scale\_noise: Callable{[}{[}Union{[}qiskit.circuit.quantumcircuit.QuantumCircuit, pyquil.quil.Program{]}, float{]}, Union{[}qiskit.circuit.quantumcircuit.QuantumCircuit, pyquil.quil.Program{]}{]} = None}}{{ $\rightarrow$ Callable{[}{[}Union{[}qiskit.circuit.quantumcircuit.QuantumCircuit, pyquil.quil.Program{]}{]}, float{]}}}
Decorator which automatically adds error mitigation to any circuit\sphinxhyphen{}executor function
defined by the user.

It is supposed to be applied to any function which executes a quantum circuit
with an arbitrary backend and produces an expectation value.
\begin{quote}\begin{description}
\item[{Parameters}] \leavevmode\begin{itemize}
\item {} 
\sphinxstyleliteralstrong{\sphinxupquote{fac}} \sphinxhyphen{}\sphinxhyphen{} Factory object determining the zero\sphinxhyphen{}noise extrapolation algorithm.
If not specified, LinearFactory({[}1.0, 2.0{]}) will be used.

\item {} 
\sphinxstyleliteralstrong{\sphinxupquote{scale\_noise}} \sphinxhyphen{}\sphinxhyphen{} Function for scaling the noise of a quantum circuit.
If not specified, a default method will be used.

\end{itemize}

\end{description}\end{quote}

\end{fulllineitems}



\section{Folding}
\label{\detokenize{apidoc:module-mitiq.folding_cirq}}\label{\detokenize{apidoc:folding}}\index{mitiq.folding\_cirq (module)@\spxentry{mitiq.folding\_cirq}\spxextra{module}}
Functions to fold gates in Cirq circuits.
\index{fold\_gates() (in module mitiq.folding\_cirq)@\spxentry{fold\_gates()}\spxextra{in module mitiq.folding\_cirq}}

\begin{fulllineitems}
\phantomsection\label{\detokenize{apidoc:mitiq.folding_cirq.fold_gates}}\pysiglinewithargsret{\sphinxcode{\sphinxupquote{mitiq.folding\_cirq.}}\sphinxbfcode{\sphinxupquote{fold\_gates}}}{\emph{circuit: cirq.circuits.circuit.Circuit, moment\_indices: Iterable{[}int{]}, gate\_indices: List{[}Iterable{[}int{]}{]}}}{{ $\rightarrow$ cirq.circuits.circuit.Circuit}}
Returns a new circuit with specified gates folded.
\begin{quote}\begin{description}
\item[{Parameters}] \leavevmode\begin{itemize}
\item {} 
\sphinxstyleliteralstrong{\sphinxupquote{circuit}} \sphinxhyphen{}\sphinxhyphen{} Circuit to fold.

\item {} 
\sphinxstyleliteralstrong{\sphinxupquote{moment\_indices}} \sphinxhyphen{}\sphinxhyphen{} Indices of moments with gates to be folded.

\item {} 
\sphinxstyleliteralstrong{\sphinxupquote{gate\_indices}} \sphinxhyphen{}\sphinxhyphen{} Specifies which gates within each moment to fold.

\end{itemize}

\end{description}\end{quote}
\subsubsection*{Examples}

(1) Folds the first three gates in moment two.
\textgreater{}\textgreater{}\textgreater{} fold\_gates(circuit, moment\_indices={[}1{]}, gate\_indices={[}(0, 1, 2){]})
\begin{enumerate}
\sphinxsetlistlabels{\arabic}{enumi}{enumii}{(}{)}%
\setcounter{enumi}{1}
\item {} 
Folds gates with indices 1, 4, and 5 in moment 0,
and gates with indices 0, 1, and 2 in moment 1.

\end{enumerate}

\begin{sphinxVerbatim}[commandchars=\\\{\}]
\PYG{g+gp}{\PYGZgt{}\PYGZgt{}\PYGZgt{} }\PYG{n}{fold\PYGZus{}gates}\PYG{p}{(}\PYG{n}{circuit}\PYG{p}{,} \PYG{n}{moment\PYGZus{}indices}\PYG{o}{=}\PYG{p}{[}\PYG{l+m+mi}{0}\PYG{p}{,} \PYG{l+m+mi}{3}\PYG{p}{]}\PYG{p}{,} \PYG{n}{gate\PYGZus{}indices}\PYG{o}{=}\PYG{p}{[}\PYG{p}{(}\PYG{l+m+mi}{1}\PYG{p}{,} \PYG{l+m+mi}{4}\PYG{p}{,} \PYG{l+m+mi}{5}\PYG{p}{)}\PYG{p}{,} \PYG{p}{(}\PYG{l+m+mi}{0}\PYG{p}{,} \PYG{l+m+mi}{1}\PYG{p}{,} \PYG{l+m+mi}{2}\PYG{p}{)}\PYG{p}{]}\PYG{p}{)}
\end{sphinxVerbatim}

\end{fulllineitems}

\index{fold\_gates\_at\_random() (in module mitiq.folding\_cirq)@\spxentry{fold\_gates\_at\_random()}\spxextra{in module mitiq.folding\_cirq}}

\begin{fulllineitems}
\phantomsection\label{\detokenize{apidoc:mitiq.folding_cirq.fold_gates_at_random}}\pysiglinewithargsret{\sphinxcode{\sphinxupquote{mitiq.folding\_cirq.}}\sphinxbfcode{\sphinxupquote{fold\_gates\_at\_random}}}{\emph{circuit: cirq.circuits.circuit.Circuit}, \emph{stretch: float}, \emph{seed: Optional{[}int{]} = None}}{{ $\rightarrow$ cirq.circuits.circuit.Circuit}}
Returns a folded circuit by applying the map G \sphinxhyphen{}\textgreater{} G G\textasciicircum{}dag G to a random subset of gates in the input circuit.

The folded circuit has a number of gates approximately equal to stretch * n where n is the number of gates in
the input circuit.
\begin{quote}\begin{description}
\item[{Parameters}] \leavevmode\begin{itemize}
\item {} 
\sphinxstyleliteralstrong{\sphinxupquote{circuit}} \sphinxhyphen{}\sphinxhyphen{} Circuit to fold.

\item {} 
\sphinxstyleliteralstrong{\sphinxupquote{stretch}} \sphinxhyphen{}\sphinxhyphen{} Factor to stretch the circuit by. Any real number in the interval {[}1, 3{]}.

\item {} 
\sphinxstyleliteralstrong{\sphinxupquote{seed}} \sphinxhyphen{}\sphinxhyphen{} {[}Optional{]} Integer seed for random number generator.

\end{itemize}

\end{description}\end{quote}

\begin{sphinxadmonition}{note}{Note:}
Folding a single gate adds two gates to the circuit, hence the maximum stretch factor is 3.
\end{sphinxadmonition}

\end{fulllineitems}

\index{fold\_gates\_from\_left() (in module mitiq.folding\_cirq)@\spxentry{fold\_gates\_from\_left()}\spxextra{in module mitiq.folding\_cirq}}

\begin{fulllineitems}
\phantomsection\label{\detokenize{apidoc:mitiq.folding_cirq.fold_gates_from_left}}\pysiglinewithargsret{\sphinxcode{\sphinxupquote{mitiq.folding\_cirq.}}\sphinxbfcode{\sphinxupquote{fold\_gates\_from\_left}}}{\emph{circuit: cirq.circuits.circuit.Circuit}, \emph{stretch: float}}{{ $\rightarrow$ cirq.circuits.circuit.Circuit}}
Returns a new folded circuit by applying the map G \sphinxhyphen{}\textgreater{} G G\textasciicircum{}dag G to a subset of gates of the input circuit,
starting with gates at the left (beginning) of the circuit.

The folded circuit has a number of gates approximately equal to stretch * n where n is the number of gates in
the input circuit.
\begin{quote}\begin{description}
\item[{Parameters}] \leavevmode\begin{itemize}
\item {} 
\sphinxstyleliteralstrong{\sphinxupquote{circuit}} \sphinxhyphen{}\sphinxhyphen{} Circuit to fold.

\item {} 
\sphinxstyleliteralstrong{\sphinxupquote{stretch}} \sphinxhyphen{}\sphinxhyphen{} Factor to stretch the circuit by. Any real number in the interval {[}1, 3{]}.

\end{itemize}

\end{description}\end{quote}

\begin{sphinxadmonition}{note}{Note:}
Folding a single gate adds two gates to the circuit, hence the maximum stretch factor is 3.
\end{sphinxadmonition}

\end{fulllineitems}

\index{fold\_gates\_from\_right() (in module mitiq.folding\_cirq)@\spxentry{fold\_gates\_from\_right()}\spxextra{in module mitiq.folding\_cirq}}

\begin{fulllineitems}
\phantomsection\label{\detokenize{apidoc:mitiq.folding_cirq.fold_gates_from_right}}\pysiglinewithargsret{\sphinxcode{\sphinxupquote{mitiq.folding\_cirq.}}\sphinxbfcode{\sphinxupquote{fold\_gates\_from\_right}}}{\emph{circuit: cirq.circuits.circuit.Circuit}, \emph{stretch: float}}{{ $\rightarrow$ cirq.circuits.circuit.Circuit}}
Returns a new folded circuit by applying the map G \sphinxhyphen{}\textgreater{} G G\textasciicircum{}dag G to a subset of gates of the input circuit,
starting with gates at the right (end) of the circuit.

The folded circuit has a number of gates approximately equal to stretch * n where n is the number of gates in
the input circuit.
\begin{quote}\begin{description}
\item[{Parameters}] \leavevmode\begin{itemize}
\item {} 
\sphinxstyleliteralstrong{\sphinxupquote{circuit}} \sphinxhyphen{}\sphinxhyphen{} Circuit to fold.

\item {} 
\sphinxstyleliteralstrong{\sphinxupquote{stretch}} \sphinxhyphen{}\sphinxhyphen{} Factor to stretch the circuit by. Any real number in the interval {[}1, 3{]}.

\end{itemize}

\end{description}\end{quote}

\begin{sphinxadmonition}{note}{Note:}
Folding a single gate adds two gates to the circuit, hence the maximum stretch factor is 3.
\end{sphinxadmonition}

\end{fulllineitems}

\index{fold\_local() (in module mitiq.folding\_cirq)@\spxentry{fold\_local()}\spxextra{in module mitiq.folding\_cirq}}

\begin{fulllineitems}
\phantomsection\label{\detokenize{apidoc:mitiq.folding_cirq.fold_local}}\pysiglinewithargsret{\sphinxcode{\sphinxupquote{mitiq.folding\_cirq.}}\sphinxbfcode{\sphinxupquote{fold\_local}}}{\emph{circuit: cirq.circuits.circuit.Circuit, stretch: float, fold\_method: Callable{[}{[}cirq.circuits.circuit.Circuit, float, Tuple{[}Any{]}{]}, cirq.circuits.circuit.Circuit{]} = \textless{}function fold\_gates\_from\_left\textgreater{}, fold\_method\_args: Tuple{[}Any{]} = ()}}{{ $\rightarrow$ cirq.circuits.circuit.Circuit}}
Returns a folded circuit by folding gates according to the input fold method.
\begin{quote}\begin{description}
\item[{Parameters}] \leavevmode\begin{itemize}
\item {} 
\sphinxstyleliteralstrong{\sphinxupquote{circuit}} \sphinxhyphen{}\sphinxhyphen{} Circuit to fold.

\item {} 
\sphinxstyleliteralstrong{\sphinxupquote{stretch}} \sphinxhyphen{}\sphinxhyphen{} Factor to stretch the circuit by.

\item {} 
\sphinxstyleliteralstrong{\sphinxupquote{fold\_method}} \sphinxhyphen{}\sphinxhyphen{} 
Function which defines the method for folding gates.
(e.g., Randomly selects gates to fold, folds gates starting from left of circuit, etc.)

Must have signature
\begin{description}
\item[{def fold\_method(circuit: Circuit, stretch: float, {\color{red}\bfseries{}**}kwargs):}] \leavevmode
...

\end{description}

and return a circuit.


\item {} 
\sphinxstyleliteralstrong{\sphinxupquote{fold\_method\_args}} \sphinxhyphen{}\sphinxhyphen{} \begin{description}
\item[{Any additional input arguments for the fold\_method.}] \leavevmode
The method is called with fold\_method(circuit, stretch, {\color{red}\bfseries{}*}fold\_method\_args).

\end{description}
\subsubsection*{Example}

fold\_method = fold\_gates\_at\_random
fold\_method\_args = (1,)

\textgreater{} Uses a seed of one for the fold\_gates\_at\_random method.


\end{itemize}

\end{description}\end{quote}

\end{fulllineitems}

\index{fold\_moments() (in module mitiq.folding\_cirq)@\spxentry{fold\_moments()}\spxextra{in module mitiq.folding\_cirq}}

\begin{fulllineitems}
\phantomsection\label{\detokenize{apidoc:mitiq.folding_cirq.fold_moments}}\pysiglinewithargsret{\sphinxcode{\sphinxupquote{mitiq.folding\_cirq.}}\sphinxbfcode{\sphinxupquote{fold\_moments}}}{\emph{circuit: cirq.circuits.circuit.Circuit, moment\_indices: List{[}int{]}}}{{ $\rightarrow$ cirq.circuits.circuit.Circuit}}
Returns a new circuit with moments folded by mapping

M\_i \sphinxhyphen{}\textgreater{} M\_i M\_i\textasciicircum{}dag M\_i

where M\_i is a moment specified by an integer in moment\_indices.
\begin{quote}\begin{description}
\item[{Parameters}] \leavevmode\begin{itemize}
\item {} 
\sphinxstyleliteralstrong{\sphinxupquote{circuit}} \sphinxhyphen{}\sphinxhyphen{} Circuit to apply folding operation to.

\item {} 
\sphinxstyleliteralstrong{\sphinxupquote{moment\_indices}} \sphinxhyphen{}\sphinxhyphen{} List of integers that specify moments to fold.

\end{itemize}

\end{description}\end{quote}

\end{fulllineitems}

\index{unitary\_folding() (in module mitiq.folding\_cirq)@\spxentry{unitary\_folding()}\spxextra{in module mitiq.folding\_cirq}}

\begin{fulllineitems}
\phantomsection\label{\detokenize{apidoc:mitiq.folding_cirq.unitary_folding}}\pysiglinewithargsret{\sphinxcode{\sphinxupquote{mitiq.folding\_cirq.}}\sphinxbfcode{\sphinxupquote{unitary\_folding}}}{\emph{circuit: cirq.circuits.circuit.Circuit}, \emph{stretch: float}}{{ $\rightarrow$ cirq.circuits.circuit.Circuit}}
Applies global unitary folding and a final partial folding of the input circuit.
Returns a circuit of depth approximately equal to stretch*len(circuit).
The stretch factor can be any real number \textgreater{}= 1.

\end{fulllineitems}



\section{Matrices}
\label{\detokenize{apidoc:module-mitiq.matrices}}\label{\detokenize{apidoc:matrices}}\index{mitiq.matrices (module)@\spxentry{mitiq.matrices}\spxextra{module}}

\section{Qiskit Utils}
\label{\detokenize{apidoc:module-mitiq.qiskit.qiskit_utils}}\label{\detokenize{apidoc:qiskit-utils}}\index{mitiq.qiskit.qiskit\_utils (module)@\spxentry{mitiq.qiskit.qiskit\_utils}\spxextra{module}}\index{random\_identity\_circuit() (in module mitiq.qiskit.qiskit\_utils)@\spxentry{random\_identity\_circuit()}\spxextra{in module mitiq.qiskit.qiskit\_utils}}

\begin{fulllineitems}
\phantomsection\label{\detokenize{apidoc:mitiq.qiskit.qiskit_utils.random_identity_circuit}}\pysiglinewithargsret{\sphinxcode{\sphinxupquote{mitiq.qiskit.qiskit\_utils.}}\sphinxbfcode{\sphinxupquote{random\_identity\_circuit}}}{\emph{depth=None}}{}
Returns a single\sphinxhyphen{}qubit identity circuit based on Pauli gates.

\end{fulllineitems}


\sphinxincludegraphics{{/Users/nathanshammah/github/mitiq/docs/build/doctrees/images/d0e7ddd334f5cd5d1e2ff83269e13074662dde9a/badge}.svg}


\chapter{Mitiq Documentation}
\label{\detokenize{README-docs:mitiq-documentation}}\label{\detokenize{README-docs::doc}}
This is the documentation of Mitiq, a Python toolkit for
implementing error mitigation on quantum computers.


\section{Requirements}
\label{\detokenize{README-docs:requirements}}
The documentation is generated with
\sphinxhref{https://www.sphinx-doc.org/en/master/usage/installation.html}{Sphinx}.

\begin{sphinxVerbatim}[commandchars=\\\{\}]
pip install \PYGZhy{}U sphinx
\end{sphinxVerbatim}


\subsection{Check your Sphinx installation}
\label{\detokenize{README-docs:check-your-sphinx-installation}}
To check that Sphinx is installed you can run

\begin{sphinxVerbatim}[commandchars=\\\{\}]
sphinx\PYGZhy{}build \PYGZhy{}\PYGZhy{}version
\end{sphinxVerbatim}


\section{How to Update the Documentation}
\label{\detokenize{README-docs:how-to-update-the-documentation}}

\subsection{Work in an environment}
\label{\detokenize{README-docs:work-in-an-environment}}\begin{itemize}
\item {} 
Create a conda environment for the documentation

\end{itemize}

\begin{sphinxVerbatim}[commandchars=\\\{\}]
conda create \PYGZhy{}n mitiqenv
conda activate mitiqenv
\end{sphinxVerbatim}


\subsection{Create a new branch}
\label{\detokenize{README-docs:create-a-new-branch}}\begin{itemize}
\item {} 
Create a branch in \sphinxcode{\sphinxupquote{git}} for the documentation with the release number up to
minor (e.g., 0.0.2\sphinxhyphen{}\sphinxhyphen{}\sphinxhyphen{}\textgreater{}00X)

\end{itemize}

\begin{sphinxVerbatim}[commandchars=\\\{\}]
\PYG{o}{(}mitiqenv\PYG{o}{)} git checkout \PYGZhy{}b mitiq00X
\end{sphinxVerbatim}


\subsection{Create a new branch}
\label{\detokenize{README-docs:id1}}\begin{itemize}
\item {} 
Since the documentation is already created, you need not to generate it
from scratch. If you had to generate it from scratch, the first step would
involve creating the \sphinxcode{\sphinxupquote{conf.py}} file. This can be generated with a wizard

\end{itemize}

\begin{sphinxVerbatim}[commandchars=\\\{\}]
\PYG{o}{(}mitiqenv\PYG{o}{)} sphinx\PYGZhy{}quickstart
\end{sphinxVerbatim}

which then asks some questions. Meta\sphinxhyphen{}data and specifications are accounted for
in the \sphinxcode{\sphinxupquote{conf.py}} file.


\subsection{Build the documentation locally}
\label{\detokenize{README-docs:build-the-documentation-locally}}\begin{itemize}
\item {} 
To build the documentation, from \sphinxcode{\sphinxupquote{bash}}, move to the \sphinxcode{\sphinxupquote{docs}} folder and run

\end{itemize}

\begin{sphinxVerbatim}[commandchars=\\\{\}]
sphinx\PYGZhy{}build \PYGZhy{}b html \PYG{n+nb}{source} build
\end{sphinxVerbatim}

this generates the \sphinxcode{\sphinxupquote{docs/build}} folder. This folder is not kept track of in the
github repository, as \sphinxcode{\sphinxupquote{docs/build}} is present in the \sphinxcode{\sphinxupquote{.gitignore}} file.
You need not to modify the \sphinxcode{\sphinxupquote{docs/build}} folder, as it is automatically
generated. You will modify only the \sphinxcode{\sphinxupquote{docs/source}} files.

The \sphinxcode{\sphinxupquote{html}} and \sphinxcode{\sphinxupquote{latex}}  and \sphinxcode{\sphinxupquote{pdf}} files will be automatically created in the
\sphinxcode{\sphinxupquote{docs/build}} folder.


\subsection{Create the html}
\label{\detokenize{README-docs:create-the-html}}\begin{itemize}
\item {} 
To create the html structure,

\end{itemize}

\begin{sphinxVerbatim}[commandchars=\\\{\}]
make html
\end{sphinxVerbatim}


\subsection{Create the pdf}
\label{\detokenize{README-docs:create-the-pdf}}\begin{itemize}
\item {} 
To create the latex files and output a pdf,

\end{itemize}

\begin{sphinxVerbatim}[commandchars=\\\{\}]
make latexpdf
\end{sphinxVerbatim}


\subsection{Add information in the guide with a tree of text files}
\label{\detokenize{README-docs:add-information-in-the-guide-with-a-tree-of-text-files}}
The documentation is divided into a guide, whose content needs to be written
from scratch, and an API doc part, which can be partly automatically generated.
\begin{itemize}
\item {} 
To add information in the guide, it is possible to include new information
as a restructured text (\sphinxcode{\sphinxupquote{.rst}}) or markdown (\sphinxcode{\sphinxupquote{.md}}) file.

\end{itemize}

The main file is \sphinxcode{\sphinxupquote{index.rst}}. It includes a \sphinxcode{\sphinxupquote{guide.rst}} and an \sphinxcode{\sphinxupquote{apidoc.rst}}
file, as well as other files. Like in LaTeX, each file can include other files.
Make sure they are included in the table of contents

\begin{sphinxVerbatim}[commandchars=\\\{\}]
\PYG{o}{.}\PYG{o}{.} \PYG{n}{toctree}\PYG{p}{:}\PYG{p}{:}
   \PYG{p}{:}\PYG{n}{maxdepth}\PYG{p}{:} \PYG{l+m+mi}{2}
   \PYG{p}{:}\PYG{n}{caption}\PYG{p}{:} \PYG{n}{Contents}\PYG{p}{:}

   \PYG{n}{changelog}\PYG{o}{.}\PYG{n}{rst}
\end{sphinxVerbatim}


\subsection{Add features in the conf.py file}
\label{\detokenize{README-docs:add-features-in-the-conf-py-file}}\begin{itemize}
\item {} 
To add specific feature to the documentation, extensions can be include.
For example to add classes and functions to the API doc, make sure that autodoc
extension is enabled in the \sphinxcode{\sphinxupquote{conf.py}} file,

\end{itemize}

\begin{sphinxVerbatim}[commandchars=\\\{\}]
\PYG{n}{extensions} \PYG{o}{=} \PYG{p}{[}\PYG{l+s+s1}{\PYGZsq{}}\PYG{l+s+s1}{sphinx.ext.autodoc}\PYG{l+s+s1}{\PYGZsq{}}\PYG{p}{]}
\end{sphinxVerbatim}


\subsection{Automatically add information to the API doc}
\label{\detokenize{README-docs:automatically-add-information-to-the-api-doc}}\begin{itemize}
\item {} 
New modules, classes and functions can be added by listing them
in the appropriate \sphinxcode{\sphinxupquote{.rst}} file (such as \sphinxcode{\sphinxupquote{autodoc.rst}} or a child), e.g.,

\end{itemize}

\begin{sphinxVerbatim}[commandchars=\\\{\}]
\PYG{n}{Factories}
\PYG{o}{\PYGZhy{}}\PYG{o}{\PYGZhy{}}\PYG{o}{\PYGZhy{}}\PYG{o}{\PYGZhy{}}\PYG{o}{\PYGZhy{}}\PYG{o}{\PYGZhy{}}\PYG{o}{\PYGZhy{}}\PYG{o}{\PYGZhy{}}\PYG{o}{\PYGZhy{}}
\PYG{o}{.}\PYG{o}{.} \PYG{n}{automodule}\PYG{p}{:}\PYG{p}{:} \PYG{n}{mitiq}\PYG{o}{.}\PYG{n}{factories}
   \PYG{p}{:}\PYG{n}{members}\PYG{p}{:}
\end{sphinxVerbatim}

will add all elements of the \sphinxcode{\sphinxupquote{mitiq.factories}} module. One can hand\sphinxhyphen{}pick
classes and functions to add, to comment them, as well as exclude them.


\subsection{Save the pdf file in the \sphinxstyleliteralintitle{\sphinxupquote{docs/pdf}} folder}
\label{\detokenize{README-docs:save-the-pdf-file-in-the-docs-pdf-folder}}
Since the \sphinxcode{\sphinxupquote{docs/build}} folder is not kept track of, copy the pdf file
with the documentation from \sphinxcode{\sphinxupquote{docs/build/latex}} to the \sphinxcode{\sphinxupquote{docs/pdf}} folder,
naming it according to the release version with major and minor.


\section{Additional information}
\label{\detokenize{README-docs:additional-information}}
\sphinxhref{https://github.com/nathanshammah/scikit-project/blob/master/5-docs}{Here}
are some notes on how to build docs.


\chapter{Indices and tables}
\label{\detokenize{index:indices-and-tables}}\begin{itemize}
\item {} 
\DUrole{xref,std,std-ref}{genindex}

\item {} 
\DUrole{xref,std,std-ref}{modindex}

\item {} 
\DUrole{xref,std,std-ref}{search}

\end{itemize}


\renewcommand{\indexname}{Python Module Index}
\begin{sphinxtheindex}
\let\bigletter\sphinxstyleindexlettergroup
\bigletter{m}
\item\relax\sphinxstyleindexentry{mitiq}\sphinxstyleindexpageref{apidoc:\detokenize{module-mitiq}}
\item\relax\sphinxstyleindexentry{mitiq.factories}\sphinxstyleindexpageref{apidoc:\detokenize{module-mitiq.factories}}
\item\relax\sphinxstyleindexentry{mitiq.folding\_cirq}\sphinxstyleindexpageref{apidoc:\detokenize{module-mitiq.folding_cirq}}
\item\relax\sphinxstyleindexentry{mitiq.matrices}\sphinxstyleindexpageref{apidoc:\detokenize{module-mitiq.matrices}}
\item\relax\sphinxstyleindexentry{mitiq.qiskit.qiskit\_utils}\sphinxstyleindexpageref{apidoc:\detokenize{module-mitiq.qiskit.qiskit_utils}}
\item\relax\sphinxstyleindexentry{mitiq.zne}\sphinxstyleindexpageref{apidoc:\detokenize{module-mitiq.zne}}
\end{sphinxtheindex}

\renewcommand{\indexname}{Index}
\printindex
\end{document}